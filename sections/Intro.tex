\section{Introduction}
%%%%%%%%%%%%%%%%%%%%%%%%%%%%%%%% FRAME 0 %%%%%%%%%%%%%%%%%%%%%%%%%%%%%%%%%%%%%%%%%%%%
\subsection{Diamond as Particle Detector}
\begin{frame}{Diamond as Particle Detector}

	\subfigs{\subfig[.5]{.5}{Detector}[Detector Schematics]}{\subfig[.4][.15]{.35}{Wafer}[\SI{15}{\centi\meter} pCVD Diamond Wafer]}
	
	\begin{itemize}\itemfill
		\item detectors operated as ionisation chambers 
		\item poly-crystals produced in large wafers
		\item metallisation on both sides
	\end{itemize}

\end{frame}

%%%%%%%%%%%%%%%%%%%%%%%%%%%%%%%% FRAME 1 %%%%%%%%%%%%%%%%%%%%%%%%%%%%%%%%%%%%%%%%%%%%
\subsection{3D Detectors}
\begin{frame}{Working Principle}

	\vspace*{-10pt}\fig{.5}{3DConcept}\vspace*{-10pt}
	
	\begin{itemize}\itemfill
		\item after large radiation fluence all detectors become trap limited
		\item bias and readout electrode inside detector material
		\item same thickness $D$ \ra same amount of induced charge \ra shorter drift distance L
		\item \good{increase collected charge in detectors with limited mean free path}
	\end{itemize}

\end{frame}

%%%%%%%%%%%%%%%%%%%%%%%%%%%%%%%% FRAME 2 %%%%%%%%%%%%%%%%%%%%%%%%%%%%%%%%%%%%%%%%%%%%
\begin{frame}{Laser drilling}

	\begin{itemize}\itemfill
		\item ``drilling'' columns with \SI{\sim15}{\micro\meter} gap to the surface using fs-laser (Oxford)
		\item convert diamond into resistive mixture of carbon phases (i.a. DLC, graphite, ...)
		\item usage of spatial light modulation (SLM) to correct for aberration
		\item initial column yield \SI{\sim90}{\%} \ra now \SI{\ge99}{\%}
		\item  initial column diameter \SIrange{6}{10}{\micro\meter} \ra now \SI{2.6}{\micro\meter}
	\end{itemize}
	
	\fig{.4}{Laser}\vspace*{-10pt}
	
\end{frame}

%%%%%%%%%%%%%%%%%%%%%%%%%%%%%%%% FRAME 3 %%%%%%%%%%%%%%%%%%%%%%%%%%%%%%%%%%%%%%%%%%%%
\begin{frame}{Bump Bonding}

	\subfigs{\subfig[.47][.1]{.3}{BondingScheme}[Bump bond schematics]}
	{\only<1>{\hspace*{3pt}\subfig[.43]{.4}{BBCMS}[\SI{3x2}{} bump pads]}
	\only<2>{\subfig[.43][.1]{.3}{BBAt}[\SI{1x5}{} bump pads]}}
	
	\begin{itemize}\itemfill
		\item connection to bias and readout with surface metallisation
		\item ganging of cells to match pixel pitch of readout-chip (ROC)
		\item small gap to the surface to avoid a break-through 
	\end{itemize}

\end{frame}

%%%%%%%%%%%%%%%%%%%%%%%%%%%%%%%% FRAME 4 %%%%%%%%%%%%%%%%%%%%%%%%%%%%%%%%%%%%%%%%%%%%
\subsection{Achievements}
\begin{frame}{Progress in Diamond Detectors}

	\textbf{\underline{3D Detectors:}}\vspace*{5pt}
	\begin{itemize}\itemfill
		\item proved that 3D works in pCVD diamond
		\item scale up the number of columns per detector: \orderof{100} \ra \orderof{1000} (x40)
		\item reducing the cell size: \SI{150x150}{\micro\meter} \ra \SI{50x50}{\micro\meter} \ra \SI{25x25}{\micro\meter} (soon)
		\item reducing the diameter of the columns: \SIrange{6}{10}{\micro\meter} \ra \SI{2.6}{\micro\meter} \ra \SIrange{1}{2}{\micro\meter} (soon)
		\item \ra increasing column yield: \SI{\sim90}{\%} \ra \SI{\ge99}{\%}
		\item recently tested first irradiated \SI{50x50}{\micro\meter} 3D detector (\SI{3.5e15}{\ncm})
	\end{itemize}
	
	\vspace*{5pt}\textbf{\underline{3D Pixel Detectors:}}\vspace*{5pt}
	\begin{itemize}\itemfill
		\item visible improvements with each step reducing the cell size
		\item all worked as expected (to first order)
	\end{itemize}
	
	\vspace*{5pt}\textbf{\underline{Rate Studies in Pad Detectors:}}\vspace*{5pt}
	\begin{itemize}\itemfill
		\item particle fluxes from \SI{1}{\khzcm} up to \SI{20}{\mhzcm}
		\item irradiations up to \SI{4e15}{\ncm}
	\end{itemize}


\end{frame}
