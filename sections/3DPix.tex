\section{3D Pixel Detectors}
%%%%%%%%%%%%%%%%%%%%%%%%%%%%%%%%%%%%%% FRAME 0 %%%%%%%%%%%%%%%%%%%%%%%%%%%%%%%%%%%%%%%%%%%%%%%%
\subsection{\SI{1x5}{} Ganging}
\begin{frame}{\SI{1x5}{} Ganging}

	\fig{.5}{3DATLAS}[Final Detector]	
	
	\begin{itemize}\itemfill
		\item readout chip (ROC): ATLAS FEI4 (\SI{50x250}{\micro\meter})
		\item Size: \SI{5x5}{\milli\meter}
		\item active area \SI{3x3}{\milli\meter}
		\item tin-silver bump bonding at IFAE (Barcelona)
	\end{itemize}
	
\end{frame}
%%%%%%%%%%%%%%%%%%%%%%%%%%%%%%%%%%%%%% FRAME 1 %%%%%%%%%%%%%%%%%%%%%%%%%%%%%%%%%%%%%%%%%%%%%%%%
\begin{frame}{Efficiencies at CERN Beam Test}

	\subfigs{\subfig{.46}{EffAtL}[High threshold (\SI{1500}{e})]}{\subfig{.46}{EffAtH}[Low threshold (\SI{1000}{e})]}
	
	\begin{itemize}\itemfill
		\item spatial resolution of \SI{\sim3}{\micro\meter}
		\item two different tunings of the FEI4 chip
		\item efficiency with low threshold significantly higher: \SI{97.7}{\%}
		\item inefficiencies most likely due to bump bonding issues
	\end{itemize}
	
\end{frame}
%%%%%%%%%%%%%%%%%%%%%%%%%%%%%%%%%%%%%% FRAME 2 %%%%%%%%%%%%%%%%%%%%%%%%%%%%%%%%%%%%%%%%%%%%%%%%
\begin{frame}{Time Over Threshold}

	\fig{.5}{SDAtH}[Time over threshold]	
	
	\begin{itemize}\itemfill
		\item \SI{5}{tot} $\approx$ \SI{11000}{e}
		\item mean of the ToT distribution: 6.73 \ra \SI{14800}{e}
		\item \SI{81}{\%} of the charge collected
	\end{itemize}
	
\end{frame}
%%%%%%%%%%%%%%%%%%%%%%%%%%%%%%%%%%%%%% FRAME 3 %%%%%%%%%%%%%%%%%%%%%%%%%%%%%%%%%%%%%%%%%%%%%%%%
\subsection{\SI{2x3}{} Ganging}
\begin{frame}{\SI{2x3}{} Ganging}

	\fig{.4}{3DCMS}[Final Detector]	
	
	\begin{itemize}\itemfill
		\item readout chip (ROC): CMS PSI46digv2.1repspin (\SI{100x150}{\micro\meter})
		\item Size: \SI{5x5}{\milli\meter}
		\item active area \SI{3.5x3.5}{\milli\meter}
		\item indium bump-bonding (Princeton)
	\end{itemize}
	
\end{frame}
%%%%%%%%%%%%%%%%%%%%%%%%%%%%%%%%%%%%%% FRAME 4 %%%%%%%%%%%%%%%%%%%%%%%%%%%%%%%%%%%%%%%%%%%%%%%%
\begin{frame}{Efficiencies - First PSI Beam Test}

	\only<1>{\subfigs{\subfig{.45}[r]{EffMap}[Efficiency Map Diamond]}{\subfig{.45}[r]{EffS}[Efficiency Map Silicon]}}
	\only<2>{\subfigs{\subfig{.45}[r]{EffMap}[Efficiency Map]}{\subfig{.45}[r]{EffVol}[Efficiency vs. voltage]}}
	
	\begin{itemize}\itemfill
		\item<1-> beam test right after the first bump bonding (top right corner badly bonded)
		\item<1-> spatial resolution of \orderof{\SI{100}{\micro\meter}}
		\only<1>{\item efficiency in red fiducial area: Diamond: \SI{99.1}{\%}, Silicon: \SI{99.9}{\%}}
		\only<2>{\item effective efficiency (relative to silicon) in red fiducial area: \SI{99.2}{\%}}
		\item<2> already fully efficient at \SI{30}{\volt}
		\item<2> ROC stopped working after this beam test
	\end{itemize}
	
\end{frame}
%%%%%%%%%%%%%%%%%%%%%%%%%%%%%%%%%%%%%% FRAME 4.1 %%%%%%%%%%%%%%%%%%%%%%%%%%%%%%%%%%%%%%%%%%%%%%%%
\begin{frame}{Efficiencies - CERN Beam Test}

	\fig{.5}{EffCern1}[Efficiency at threshold of \SI{\sim3500}{e}]
	
	\begin{itemize}\itemfill
		\item high resolution measurement at CERN
		\item find non-working/non-connected cells
		\item sensor twice re-bump-bonded with the same indium (no reprocessing)
		\begin{itemize}
			\item no removal of old bumps, no change of surface metallisation
		\end{itemize}
	\end{itemize}
	
\end{frame}
%%%%%%%%%%%%%%%%%%%%%%%%%%%%%%%%%%%%%% FRAME 5 %%%%%%%%%%%%%%%%%%%%%%%%%%%%%%%%%%%%%%%%%%%%%%%%
\begin{frame}{Efficiencies - Second PSI Beam Test}

% 	\only<1>{\subfigs{\subfig{.45}[r]{EffMap2}[Efficiency Map Diamond]}{\subfig{.45}[r]{EffS2}[Efficiency Map Silicon]}}
	\subfigs{\subfig{.45}[r]{EffMap2}[Efficiency Map]}{\subfig{.45}[r]{EffVol2}[Efficiency vs. voltage]}
	
	\begin{itemize}\itemfill
		\item sensor twice re-bump-bonded with the same indium (no reprocessing)
% 		\only<1>{\item efficiency in red fiducial area: Diamond: \SI{97.3}{\%}, Silicon: \SI{100.0}{\%}}
		\only{\item effective efficiency in red fiducial area: \SI{97.3}{\%}}
		\item already fully efficient at \SI{30}{\volt}
		\item only very small area working well \ra many bump bond problems
	\end{itemize}
	
\end{frame}
%%%%%%%%%%%%%%%%%%%%%%%%%%%%%%%%%%%%%% FRAME 2 %%%%%%%%%%%%%%%%%%%%%%%%%%%%%%%%%%%%%%%%%%%%%%%%
\begin{frame}{Pulse Height - Second Beam Test}

	\subfigs{\subfig{.45}[r]{SD}[Signal Distribution]}{\subfig{.45}[r]{PHV}[Pulse height vs. voltage]}
	
	\begin{itemize}\itemfill
		\item wrong pulse height calibration in first beam test
		\item full charge collection also at \SI{30}{\volt}
		\item mean pulse height: \SI{11000}{e} \ra $\simeq$\SI{14000}{e} at CERN \ra consistent with \SI{1x5}{} data
% 		\item low pulse height not understood
	\end{itemize}
	
\end{frame}
